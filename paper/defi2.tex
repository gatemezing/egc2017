%% -*- TeX -*- -*- FR -*-

%Avant propos : ces exemples de fichiers ont été mis à jour grâce à
%l'aide précieuse de Gilbert Ritschard. Pour toute question ou
%remarque n'hésitez pas à nous contacter : venturin@univ-tours.fr ou
%gilbert.ritschard@themes.unige.ch
%Version 2005-10-01

%\documentclass[a4paper,french]{rnti}
%\documentclass[a4paper,french,noresume]{rnti} %% Pour papier de 2 pages

%%% Avec l'option "showlayout" vous obtenez les deux pages
%%% de contrôle des paramètres de mise en page.

\documentclass[a4paper,pagenum,french,showlayout]{rnti}


\usepackage{graphicx}
%packages nécessaires pour écrire des articles en français en utilisant les accents non latex.
\usepackage[T1]{fontenc}
\usepackage[latin1]{inputenc}

%pour bien présenter les URL et autres adresses emails
\usepackage{url}

% Titre court pour entête
\titrecourt{GeoCarteApp: Un outil générique et personnalisable pour la visualisation de données géospatiales}

% Noms auteurs pour entête :
%    Si un seul auteur, mettre : Initiale. NomPremierAuteur
%    Si deux auteurs, mettre : Initiale1. NomPremierAuteur et Initiale1. NomDeuxiemeAuteur
%    Si plus de deux auteurs, mettre comme ci-dessous
%
\nomcourt{I. NomPremierAuteur et al.}


\titre{G\'eoCarteApp: Un outil g\'en\'erique et personnalisable pour la visualisation de donn\'ees g\'eospatiales}

\auteur{Noorani Bakerally\affil{1},
        Ghislain Atemezing\affil{2}\\
        Antoine Zimmermann\affil{1}}

\affiliation{
    \affil{1}Univ Lyon, MINES Saint-\'Etienne, CNRS, Laboratoire Hubert Curien UMR 5516, \\F-42023 Saint-\'Etienne, France\\
          \{prenom.nom\}@emse.fr\\
    %
    \affil{2}Mondeca, \\ 35 boulevard Strasbourg, Paris, France\\
          ghislain.atemezing@mondeca.com\\
          %\http{http://www.mondeca.com}
 }

\resume{%
GeoSpatial data is becoming more and more important in numerous domains and available on the Web. Many standards related to geospatial data such as GeoSPARQL or GML have been defined to facilitate interchange, reasoning and querying geodata on the Web. However, there are still many datasets provided by organisations and open data portals which provide unstructured geospatial data. However, there a missing tools which can help geospatial domain experts to bring different types of geospatial data, whether standardised, partially standard or heterogeneous data  on a single map for analysis. In this paper, we provide a geospatial tool, GéoCarteApp, a generic web application which can allow the visualisation of geospatial data irrespective of the data model or format through the use of personalized configurations. In this paper, we describe the tool and show its usefulness for the trees' dataset related to Grenoble. GéoCarteApp is based on a set of configurations that enables the users to visualize geospatial datasets in heterogeneous formats.
%
}

\summary{%
GeoSpatial data is becoming more and more important in numerous domains and available on the Web. Many standards related to geospatial data such as GeoSPARQL or GML have been defined to facilitate interchange, reasoning and querying geodata on the Web. However, there are still many datasets provided by organisations and open data portals which provide unstructured geospatial data. However, there a missing tools which can help geospatial domain experts to bring different types of geospatial data, whether standardised, partially standard or heterogeneous data  on a single map for analysis. In this paper, we provide a geospatial tool, GéoCarteApp, a generic web application which can allow the visualisation of geospatial data irrespective of the data model or format through the use of personalized configurations. In this paper, we describe the tool and show its usefulness for the trees' dataset related to Grenoble. GéoCarteApp is based on a set of configurations that enables the users to visualize geospatial datasets in heterogeneous formats.
%
}


\begin{document}
% DEBUT DE L'ARTICLE
%
\section{Introduction}

Pour produire un article en \LaTeX\ nous recommandons d'utiliser la
classe \textit{rnti.cls} version~3. Voir également le guide
d'utilisation de la classe \citep{ritschard:2005rnti}. Merci de
vérifier qu'il ne subsiste pas d'\emph{overfull box} (dépassement
des marges).

Pour les textes de 2 pages ou moins (posters), ne pas mettre de
résumé au début. On donnera cependant un ``summary'' (en anglais)
d'au plus 40 mots en fin de texte. Utiliser pour cela l'option
\textit{noresume} qui supprime le résumé mais maintient le summary.

\section{GéoCarteApp: Approche Générique}

\section{Application aux données de Grenoble}

\subsection{Transformation et Enrichissement des données}

\subsection{Transformation}

\subsection{Enrichissement}


\section{Cas d'usage}

\section{Préconisations sur les données}


\subsubsection{Section de niveau 3}

Exemple de section de niveau 3. Il vaut mieux éviter, dans la mesure
du possible, d'utiliser ce niveau de section surtout pour les
articles courts. Utiliser de préférence
\verb+\paragraph{+\textit{Titre du paragraphe.}\verb+}+ qui produit

\paragraph{Titre du paragraphe.} Ceci est un exemple de section type
paragraphe. On note l'absence de numérotation.

\subsection{Liste d'éléments}

Merci de respecter les styles suivants pour les listes.

Exemple de liste sans numérotation :
\begin{itemize}
    \item premier item ;
    \item deuxième item ;
    \begin{itemize}
      \item premier sous-item,
      \item second sous-item,
      \item dernier sous-item.
  \end{itemize}
  \item dernier item.
\end{itemize}

Autre liste possible avec numérotation :
\begin{enumerate}
    \item premier item ;
    \item deuxième item ;
    \begin{enumerate}
      \item premier sous-item,
      \item second sous-item,
      \item dernier sous-item.
    \end{enumerate}
  \item dernier item.
\end{enumerate}

\subsection{Tables, Figures}


Les tableaux et figures doivent être centrés horizontalement. Les
légendes sont placées sous ces mêmes éléments et sont centrées si
elle font moins d'une ligne et justifiées sinon. Voir le
tableau~\ref{tab_exemple} et la figure~\ref{fig_exemple} pour des
exemples. Les légendes doivent se terminer avec un point.



\begin{table}[ht]
 \begin{center}
   \tabcolsep = 2\tabcolsep
   \begin{tabular}{lcc}
   \hline\hline
                & modèle 1 & modèle 2 \\
   \hline
   indicateur 1 & 1        & 2        \\
   indicateur 2 & 2        & 1        \\
   indicateur 3 & 1        & 1        \\
   \hline
   \end{tabular}
\caption{Exemple de tableau avec sa légende.} \label{tab_exemple}
 \end{center}
\end{table}

Pour les tableaux, éviter de multiplier les lignes séparatrices de
colonnes et de lignes, cela alourdi inutilement la page.


\begin{figure}[t]
\begin{center}
% \includegraphics[width=6cm]{fg_complex}
 \caption{Exemple de figure avec sa légende. La légende est centrée,
sauf si, comme ici, elle fait plus d'une ligne, auquel cas elle est
justifiée.} \label{fig_exemple}
\end{center}
\end{figure}

N'oubliez pas, le cas échéant, d'indiquer la signification des axes
sur vos graphiques.


\section{Conclusion et Perspectives}

Le texte de la note apparaît en bas de la page.\footnote{Exemple de
note de bas de page.}

\subsection{Noms des auteurs et titre court dans l'entête}

Pour les noms des auteurs utilisés dans l'entête de l'article, merci
de faire figurer un ou deux noms (avec initiales des prénoms) pour
des articles à un ou deux auteurs (``Initiale. NomPremierAuteur'',
ou ``Initiale1. NomPremierAuteur et Initiale2. NomDeuxièmeAuteur''),
et pour les articles à trois auteurs ou plus, utiliser une notation
du type ``Initiale. NomPremierAuteur et al.''. Seule la première
lettre des noms doit être en capitale, le reste est en minuscules.
De même, le titre court doit être en minuscules.



\subsection{Pas de pagination, page A4 et article final au format PDF}

Ne pas indiquer la numérotation des pages de votre article car
celle-ci sera ajoutée lors de la fabrication du numéro.

Dans la version soumise pour évaluation, il est souhaitable
cependant de numéroter les pages. Ceci facilite les commentaires des
relecteurs.

L'article final doit être impérativement remis au format PDF avec
une dimension de page A4 (la taille letter n'est pas admise).
L'option \texttt{a4paper} produit du format A4 seulement si vous
compilez avec pdf\LaTeX{}. Si vous produisez un .dvi ou .ps que vous
covertissez ensuite en pdf avec par exemple dvi2pdf ou ps2pdf, il
vous appartient de configurer correctement votre convertisseur ou de
lui fournir l'option appropriée (-pagesize=A4).


\section{Consignes pour les références}

Les références sont données en fin d'article avant le
\guilo{}Summary\guilf{}. Elles doivent être listées par ordre
alphabétique.  Nous vous recommandons l'utilisation de
\textsc{Bib}\TeX\ avec le style bibliographique \textit{rnti.bst}
décrit dans \citet{ritschard:2005rnti}. Si vous n'utilisez pas ce
style, merci de suivre les exemples donnés à la fin de cet exemple
d'article.

Dans le corps du texte, on utilise \citet{thSau00}, \citet{HolWil90}
pour faire référence à un article avec un ou deux auteurs et
\citet{lioni01} lorsque trois auteurs ou plus sont présents.  Selon
les cas, on pourra aussi utiliser \citep{thSau00} ou
\citep[voir][chapitre 4]{brei:frie:olsh:ston:1984}, ou encore
\citet{quin:1986ID3,quin:1993} dans le cas de citations multiples
d'un même auteur. Merci de suivre ces exemples.



\bibliographystyle{rnti}
\bibliography{biblioaz}

\appendix
\section*{Annexe}

Voici un exemple d'annexe. S'il y a plus de deux annexes, merci de
les numéroter (Annexe~1, Annexe~2, etc).

\Fr


\end{document}
