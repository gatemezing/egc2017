\documentclass[a4paper,pagenum,english]{rnti}
\usepackage{graphicx}
\usepackage[T1]{fontenc}
\usepackage[latin1]{inputenc}
\usepackage{url}
\titrecourt{GeoCarteApp: A generic and personalized tool for the visualization of geospatial data}

\nomcourt{I. NomPremierAuteur et al.}


\titre{GeoCarteApp: A generic and personalized tool for the visualization of geospatial data}

\auteur{Noorani Bakerally\affil{1},
        Ghislain Atemezing\affil{2}\\
        Antoine Zimmermann\affil{1},
        Olivier Boissier\affil{1}}

\affiliation{
    \affil{1}Univ Lyon, MINES Saint-\'Etienne, CNRS, Laboratoire Hubert Curien UMR 5516, \\F-42023 Saint-\'Etienne, France\\
          \{prenom.nom\}@emse.fr\\
    %
    \affil{2}Mondeca, \\ 35 boulevard Strasbourg, Paris, France\\
          ghislain.atemezing@mondeca.com\\
          %\http{http://www.mondeca.com}
 }

\resume{%
GeoSpatial data is becoming more and more important in numerous domains. Many standards related to geospatial data such as GeoSPARQL or GML have been defined to facilitate interchange, reasoning and querying of geospatial data on the Web. However, there are still many datasets provided by organizations and open data portals which provide unstructured geospatial data. However, there is a need for such a tool which can help domain experts to bring different types of geospatial data, whether standardized, partially standard or heterogeneous data on a single map for analysis. In this paper, we provide a tool, GeoCarteApp, which can allow the visualization of geospatial data irrespective of the data model or format through the use of default or personalized configurations. We describe the tool, its features and show its usefulness for the trees' dataset related to Grenoble%
}



\begin{document}
% DEBUT DE L'ARTICLE
%
\section{Introduction}
- Spatial data available
- programatically process to display data on map applications
- a need to have a generic approach to display data on map
- as of now, map application already allow drag and drop for data in GeoJSON
- possible only because data is in a standard and expected data model
- this features works for homogenous data
- as of now, many organisations still have their data in different format (RDF,CSV,JSON,...)
- How can the Semantic Web help in standardising or semi-standardising heterogenous spatial data
	- many standard have already been defined related spatial data
		- within the Semantic Web community, some standards and best practices for structuring spatial data are already available such as GeoSPARQL or WGS84 vocabulary 
- reference for data format % on the open data portals
- GeoJSON lacks expressivity, RDF is required,
- Open data, not GeoJSON, inclusion of SPARQL Generate,

- what about heterogenous data ?
- for this to work for heterogenous data, there is a need for two things:
	- an ontology for describing a dataset so that its content can be shown on a map
	- a processor which can consume data structured as per that ontology and standardise or semi-standardise 
	data so that map application can unambigiously use it for visualisation
- is there any work done on this previously, the ontology ?
- In this paper:
	- describe the ontology 
	- describe how the processor works
	
	- provide an instance of an application which uses the processor to visualize data from heterogenous sources
	- how can such appliaction be used for Grenoble data about trees

- Map application does not allow customising data from GeoJSON
- Does not allow data from heterogenous sources: RDF, CSV, JSON
- Our approach allows data from GeoJSON as well as RDF and the architecture is such that data from heterogenous sources such as CSV, JSON or even HTML can be considered
\section{ GeoCarteApp: A Generic Approach}
---------------
- architecture
- displaying RDF data on the map
- format ontology
- filtering mechanism
- SPARQL Generate for other heterogenous data
- - Using the generic approach, the application displays data on the map using a default approach, where unique colors generated for icons,
- filters, labels raw, filter options raw labels

- Architecture diagram + possibilities of SPARQL Generate
- Pop up data items:
	- key value pairs

- labels 
- asssociate class names
- specify pop up data items
	- labels for different data items
- different types of constraints
\section{Customisation/Configuration (source) }
- Application can show:
	- RDF dataset
		- need to specify url of RDF dataset
		- by default it uses WGS84 ontology to identify latitude and longitude
		- as of now, works only for Marker layer
		- can be further generalised with GeoSPARQL ontology to other types of vector layer such as polygon,.. 
	- SPARQL Results
		- need to specify:
			- SPARQL endpoint
			- SPARQL Query
			- one solution for each marker
			- need to specify latitude and longitude
			- as of now, working only for Marke Layer
	- GeoJSON data by default
		- specify URL of GeoJSON file
		- works for both Marker and Vector Layer

- Default Options
	- for each layer, there is a default unique color
	- a default description can be provided
		- where the popup include a popup with all the property and values for that layer
	- Filters to be able to search for specific layers based on property and values

Personalisation (Visualisation Part)
-------------------------------------
	- layer
		- derived attributes
		- specific customization for item with specific attribute
			- as of now, different icons for different markers having a particular attributes or attribute values
		- can customize the default desription and include in it property values of the layer
	- Marker
		- specify color
		- specify icon
	- constraints
		- area constraints
			- sometimes, not possible to show all markers
			- can specifiy section where the markers should be restricted
			- multiple sectors can be specified then, each marker is shown only if it is in all sections

	- filter labels

- as of now, no user interface to generate configuration, in an ideal situation, a user unaware of technicalities could use data from any source to generate map mashups

\section{Application aux données de Grenoble}

\subsection{Transformation et Enrichissement des données}

\subsection{Transformation}
- ontologies used ?

\subsection{Enrichissement}
	- not yet ?

\section{Cas d'usage}
- search for trees which have to be diagnose
- specific icon for trees with allergies
- others ?

- Species which can create allergies
- schools + hospital + allergy
- Species from DBPedia
- frequentation cible + allergy
\section{Préconisations sur les données}




\section{Conclusion et Perspectives}

- the importance of a data model to describe data generically so that it can be used by map application
	- Perspective
		- enhance genericity
			- constraints ?
	- propose an ontology for such configuration
	- a definition of the processor to the processing of the dataset
	- a user interface to allow users to create their own configuration	
	- other data formats
	

\bibliographystyle{rnti}
\bibliography{biblioaz}

\end{document}
